\chapter{Conclusion}

Ce travail a permis de mettre en place l’infrastructure nécessaire à l’exploitation d’un capteur QCM dans le cadre d’expériences de caractérisation de fluides et de gels. Le développement du logiciel de communication et de traitement a constitué une étape clé, en fournissant une interface personnalisée pour le pilotage des mesures, la visualisation en temps réel des données et le stockage structuré des résultats. Grâce à ce système, il est désormais possible de mesurer plusieurs harmoniques de résonance, de contrôler finement les paramètres d’acquisition, de mesurer la température interne du capteur, et d’analyser l’influence de divers milieux sur sa réponse fréquentielle.

Les expérimentations menées sur des liquides ont révélé un lien entre la viscosité et les variations de fréquence ainsi que d’amplitude, ce qui suggère que le capteur QCM peut être utilisé pour quantifier la viscosité des fluides.

Cependant, certaines limites subsistent. Le modèle utilisé ne concorde pas avec ceux présents dans la littérature. Des mesures supplémentaires seraient nécessaires pour mieux comprendre les phénomènes observés. Par ailleurs, l’ajout d’un système de régulation thermique permettrait un meilleur contrôle de la température du capteur, ce qui améliorerait la fiabilité des mesures.

Le projet peut être considéré comme un succès : les objectifs de développement et de mesure ont été atteints, le logiciel fonctionne correctement, un système de contrôle et de mesure de la température a été mis en place, et les résultats obtenus ouvrent la voie à de futures recherches

\vfil
\hspace{8cm}\makeatletter\@author\makeatother\par
\hspace{8cm}\begin{minipage}{5cm}
    % Place pour signature numérique
    \printsignature
\end{minipage}