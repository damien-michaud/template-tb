\chapter{Conclusion}

Ce travail intermédiaire a permis de mettre en place l’infrastructure nécessaire pour exploiter un capteur QCM dans le cadre d’expériences de caractérisation de fluides. Le développement du logiciel de communication et de traitement a été une étape clé, offrant une interface personnalisée pour le pilotage des mesures, la visualisation en temps réel des données et le stockage structuré des résultats. Grâce à ce système, il est désormais possible de mesurer plusieurs harmoniques de résonance, de contrôler finement les paramètres d’acquisition et d’analyser l’influence de divers milieux sur la réponse fréquentielle du quartz.

Les premières expérimentations menées sur des liquides de viscosité connue ont montré que la fréquence de résonance et l’amplitude varient sensiblement en fonction du fluide, confirmant la sensibilité du QCM à des paramètres physiques tels que la viscosité. De plus, les mesures de gélification de la gélatine bovine ont mis en évidence des transitions de phase détectables via les changements de fréquence en fonction de la température, démontrant le potentiel de cette méthode pour le suivi en temps réel de phénomènes physico-chimiques.

Cependant, plusieurs limites restent à explorer. La corrélation précise entre la viscosité et les paramètres mesurés nécessite un plus grand nombre de points de mesure et un meilleur contrôle de la température. De plus, l’étalonnage du système devra être approfondi, notamment par comparaison avec des instruments de référence.

Les prochaines étapes consistent à mesurer la viscosité à l’aide d’un viscosimètre à différentes températures, ainsi qu’à analyser la réponse fréquentielle des échantillons en fonction de la température en utilisant le système développé pour l’étude de la gélatine. L’utilisation d’un capteur de température immergé dans le liquide devrait permettre d’obtenir des résultats plus précis et représentatifs.




% La signature est optionnelle et pas forcément nécessaire...
\vfil
\hspace{8cm}\makeatletter\@author\makeatother\par
\hspace{8cm}\begin{minipage}{5cm}
    % Place pour signature numérique
    \printsignature
\end{minipage}