\chapter{Code Arduino PT100}

\begin{lstlisting}[language=C++, caption={Arduino code for PT100 temperature sensor}]
const int sensorPin = A0;  // Analog pin connected to PT100 circuit
const float Vref = 5.0;    // Reference voltage (typically 5V or 3.3V)
const int ADCmax = 1023;   // 10-bit ADC

void setup() {
    Serial.begin(9600);  // Start Serial communication
}

void loop() {
    if (Serial.available() > 0) {  // Check if data is available in Serial
        String command = Serial.readStringUntil('\n'); // Read the command
        command.trim();  // Remove any leading/trailing spaces

        if (command == "GET_TEMP") {  // If command matches expected input
            int rawValue = analogRead(sensorPin); // Read raw ADC value
            float voltage = (rawValue / (float)ADCmax) * Vref; // Convert to voltage
            float temperature = convertVoltageToTemperature(voltage); // Convert to temperature

            Serial.println(temperature, 2); // Send temperature as response
        }
    }
}

// Replace with your own conversion formula based on your circuit
float convertVoltageToTemperature(float voltage) {
    // Example: linear approximation
    return voltage * 22.97 - 5.90;
}
\end{lstlisting}