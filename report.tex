\documentclass[
    iai, % Saisir le nom de l'institut rattaché
    eai, % Saisir le nom de l'orientation
    %confidential, % Décommentez si le travail est confidentiel
]{heig-tb}

%% Choix de la police de caractères
%% (décommenter si vous souhaitez une police sans serif)
%\setmainfont{TeX Gyre Heros} % Sans sérif

%% Packages supplémentaires si besoin
\usepackage{lipsum} % Générer du texte de remplissage pour les exemples
\usepackage{parcolumns} % Si besoin de gérer des colonnes parallèles
\usepackage{tikzducks} % Utile pour générer des images temporaires
\usepackage{chemfig} % Si besoin d'insérer des formules chimiques
\usepackage[version=4]{mhchem} % Si besoin d'insérer des formules chimiques
\usepackage{graphicx}
\usepackage[
  nooldvoltagedirection,
  european,
  americaninductors]{circuitikz} % Pour les schémas électriques

% Si vous voulez générer des graphiques avec pgfplots
\usepackage{pgfplots}
\usepackage{pdfpages}
\usepgfplotslibrary{colormaps}
\pgfplotsset{width=10cm,compat=1.18}

% Types de théorèmes
\newtheorem{theorem}{Théorème}[section]
\newtheorem{corollary}{Corollaire}[theorem]
\newtheorem{lemma}[theorem]{Lemme}

\signature{signature.svg} % Remplacer par votre propre signature vectorielle.

\makenomenclature
\makenoidxglossaries
\makeindex

\addbibresource{bibliography.bib}

% Notez vos acronymes ici, par exemple:

\newacronym{qcm}{QCM}{Microbalance à cristal de quartz}
\newacronym{adc}{ADC}{Convertisseur analogique-numérique}
\newacronym{pt100}{PT100}{Thermomètre à résistance platine}


\input{backmatter/glossary}
% Auteur du document (étudiant-e) en projet de Bachelor
\author{Damien Michaud}

% Activer l'option pour l'accord du féminin dans le texte
%\setmale
\setfemale

% Titre de votre travail de Bachelor
\title{Capteurs piézoélectriques à base de nanofibres: adaptation d'une microbalance à quartz}

% Le sous titre est optionnel
\subtitle{Travail de Bachelor}

% Nom du professeur responsable
\teacher {Prof. S. Schintke (HEIG-VD)}

% Mettre à jour avec la date de rendu du travail
\date{\today}

% Numéro de TB
\thesis{7212}


\surroundwithmdframed{minted}

% Utilisé par la nomenclature
\renewcommand\nomgroup[1]{%
  \item[\bfseries
              \ifstrequal{#1}{A}{Constantes physiques}{%
                \ifstrequal{#1}{B}{Groupes}{%
                  \ifstrequal{#1}{C}{Autres Symboles}{}}}%
        ]}

\newcommand{\nomunit}[1]{%
  \renewcommand{\nomentryend}{\hspace*{\fill}#1}}


\renewcommand{\appendixname}{Annexe}
\renewcommand{\appendixtocname}{Annexes}
\renewcommand{\appendixpagename}{Annexes}

%% Début du document
\begin{document}

\selectlanguage{french}
\maketitle
\frontmatter
\clearemptydoublepage

% Ajout de l'inclusion du frontmatter
\inputdir[order=natural]{frontmatter}

%% Sommaire et tables
\clearemptydoublepage
{
  \tableofcontents
}

\printnomenclature
\clearemptydoublepage
\pagenumbering{arabic}
\mainmatter

\inputdir[order=natural]{content}

\clearpage
\printbibliography

  \listoffigures
  \let\cleardoublepage\clearpage
  \listoftables
  \let\cleardoublepage\clearpage

\printnomenclature
\clearemptydoublepage
\pagenumbering{arabic}
\chapter*{Remerciements}
\addcontentsline{toc}{chapter}{Remerciements}

Je tiens à exprimer ma gratitude au Prof. Dr. Silvia Schinke, M. Stéphane Del Rossi, Dr. Pierre-Alain Michaud, M. Samuel Reishteiner et M. Alex Yennis pour leur soutien, leurs conseils et leur accompagnement tout au long de ce travail.
\appendix
\appendixpage
\addappheadtotoc

\inputdir[order=natural]{appendix}

\let\cleardoublepage\clearpage
\backmatter

\label{glossaire}
\printnoidxglossary

\clearpage
\Large\textbf{Colophon :}\par\normalsize
\thispagestyle{empty}
La qualité de cet ouvrage repose que le moteur \LaTeX. La mise en page et le format sont inspirés d'ouvrages scientifiques tels que le modèle de thèse de l'EPFL et celui des publications O'Reilly.

Les diagrammes et les illustrations sont édités depuis l'outil en ligne draw.io. Certaines illustrations ont été reprises dans Adobe Illustrator. Les représentations 3D sont exportées de Fusion 360 et certains graphiques sont générés depuis un code source Python.



Ce document a été compilé avec \mbox{Lua\TeX}.

La famille de police de caractères utilisée est \emph{Computed Modern} créée par Donald Knuth avec son logiciel METAFONT.

\vfil




\nomenclature[U]{F}{Farad (unité de capacité)}
\nomenclature[U]{V}{Volt (unité de tension)}
\nomenclature[U]{\(\Omega\)}{Ohm (unité de résistance)}
\nomenclature[U]{dB}{Décibel (unité logarithmique)}

\nomenclature[C]{\(\eta\)}{Viscosité dynamique}
\nomenclature[C]{\(\rho\)}{Masse volumique}


\end{document}
